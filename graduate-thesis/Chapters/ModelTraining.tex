\chapter{Details of System}
\label{ch:modelTrainingProcess}

This chapters shows the details of whole system, including parameters of all training and transforming algorithm, and processing of each part.

\section{Detail Parameters}
Details of the parameters can be found in Table~\ref{tb:parameters}

\begin{table}[h]
	\centering
	\begin{tabular}{|l|l|l|}
		\hline
		\multicolumn{1}{|c|}{\textbf{Algorithm}} & \multicolumn{1}{c|}{\textbf{Parameters}} & \multicolumn{1}{c|}{\textbf{Optimize Method}} \\ \hline
		\textbf{\begin{tabular}[c]{@{}l@{}}Random Forest\\ (Classifier \& Regressor)\end{tabular}} & \begin{tabular}[c]{@{}l@{}}Trees Number: 40\\ Max Depth: 20\end{tabular} & NA \\ \hline
		\textbf{SVM} & \begin{tabular}[c]{@{}l@{}}Iterations: 100000\\ Learning rate: 0.001\end{tabular} & SGD \\ \hline
		\textbf{Logistic Regression} & \begin{tabular}[c]{@{}l@{}}Iterations: 100000\\ Learning rate: 0.001\end{tabular} & SGD \\ \hline
		\textbf{Linear Regression} & \begin{tabular}[c]{@{}l@{}}Iterations: 100000\\ Learning rate: 0.001\end{tabular} & SGD \\ \hline
		\textbf{ANN} & \begin{tabular}[c]{@{}l@{}}Iterations: 15\\ Learning rate: 0.0001\end{tabular} & BGD \\ \hline
	\end{tabular}
	\caption{Parameters of Every learning method}
	\label{tb:parameters}
\end{table}


For the normalization method, the range of min-max normalizer is $ [-1, 1] $.\\

The PCA transformer follows the default configuration in sklearn.\\

The neural network is a four-layer (two hidden layer) model. A two-layer model is just same as linear regression. The reason for choosing 2 hidden layers is that "can represent an arbitrary decision boundary to arbitrary accuracy with rational activation functions and can approximate any smooth mapping to any accuracy"\cite[p.~128]{heaton2008introduction}.\\


To determine the number of neurons use in the hidden layers, Jeff Heaton also wrote in \cite[p.~129]{heaton2008introduction} that "The number of hidden neurons should be between the size of the input layer and the size of the output layer". So, in this study, the nodes number of the first hidden layer is $ \frac{2}{3} $ of that of the input layer, and that of the second is $ \frac{1}{3} $ of that of the input layer.

\section{Data Mining Processing}
The data mining process in this system include data preprocessing, data normalization, PCA transformation, model training and price prediction. The technique details about data process are introduced in Chapter~\ref{ch:mining}. Data clean process just follows the rules introduced in section~\ref{sec:dataClean}, let's start with the transformation process.

\subsection{Data Transformation (Normalization and PCA)}

Feature vectors transform process are,
\begin{enumerate}
	\item Train a data normalizer (Min Max Normalization) and translate training features into normalized data.
	\item Train a PCA transformer using step 1's output, and transform those data.
	\item Train a second data normalizer (also Min Max Normalization) and translate step 2's output into processed data.
	\item Use the same transformer in step 1-3 to transform testing features step by step.
\end{enumerate}

The label data process only handles training data, there are two types of transform method based on the model used in next step,
\begin{itemize}
	\item If training model is single learning algorithm, i.e., the model are used to predict the stock price directly. Then the label is normalized based on the high and low price of last five days,
	\begin{equation}
	label'=\frac{label*2-(high + low)}{high - low}
	\end{equation}
	\item If training model is the combine of two learning algorithm, i.e., one is used to predict change direction, the other is to forecast amount. The trend label is transformed with,
	\begin{equation}
	trend=\begin{cases}
	1 & \text{ if } price_{tomorrow} > price_{today} \\ 
	0 & \text{ if } otherwise
	\end{cases}
	\end{equation}
	The amount label is calculated by,
	\begin{equation}
	amount=\left | price_{tomorrow} - price_{today} \right |
	\end{equation}
	and after the above calculation, the amount value will be normalized using standard normalization algorithm. 
\end{itemize}

After the above process, data is transformed and can be used for model training.

\subsection{Model Training and Predicting process}
As described before, there are two types of combination of learning methods used in this dissertation. 
\begin{itemize}
	\item Using one single algorithm to predict the stock price directly. Such algorithms include Linear Regression, Random Forest and ANN
	\item A combination of two learning method, using one algorithm to predict change direction (Logistic Regression, Random Forest and SVM), another to forecast amount (Linear Regression, Random, Forest and ANN). The two method works separately, and the predict result is the combination of both model.
\end{itemize}

These two learning methods has very small difference in the training and predicting process, 
\begin{enumerate}
	\item Randomly split  
\end{enumerate}

