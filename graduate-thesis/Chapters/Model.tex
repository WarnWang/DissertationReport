\chapter{System Design and Evaluation Method}
\label{ch:system}

The objective of this dissertation is to build an automatic system which can download raw data, do transformation, train model, and give prediction result. This system is based on Python, a Friendly power programming language. “Quandl”, “TA-Lib”, “SciPy”, “pySpark”, “SciPy-learner” are used for data collecting, mining analysis and handling times series data structure. The whole process can be found in figure~\ref{fg:system_model}.
\begin{figure}[h]
	\centering
	\includegraphics[width=0.6\textwidth]{Model/FlowChart}
	\caption{Flow Chart of the whole system}
	\label{fg:system_model}
\end{figure}

After collected all raw data, system would split it into two parts, training and testing (details about collected data is introduced in Chapter\ref{ch:market}). To make an convincing experiment, \emph{the date of all testing data is behind that of training data}, so that the system would have no knowledge of the testing data. Let's talk about the details of data processing and model training process next.\\


\section{Data Processing}
The data processing part include data normalization and PCA. Feature vectors of testing and training data, labels of training data should be normalized, technique details are introduced in Chapter~\ref{ch:mining}.\\


Feature vectors transform process are,
\begin{enumerate}
	\item Train a data normalizer (Min Max Normalization) and translate training features into normalized data.
	\item Train a PCA transformer using step 1's output, and transform those data.
	\item Train a second data normalizer (also Min Max Normalization) and translate step 2's output into processed data.
	\item Use the same transformer in step 1-3 to transform testing features step by step.
\end{enumerate}

The label data process only handles training data, there are two types of transform method based on the model used in next step,
\begin{itemize}
	\item If training model is single learning algorithm, i.e., the model are used to predict the stock price directly. Then the label is normalized based on the high and low price of last five days,
	\begin{equation}
		label'=\frac{label*2-(high + low)}{high - low}
	\end{equation}
	\item If training model is the combine of two learning algorithm, i.e., one is used to predict change direction, the other is to forecast amount. The trend label is transformed with,
	\begin{equation}
	trend=\begin{cases}
	1 & \text{ if } price_{tomorrow} > price_{today} \\ 
	0 & \text{ if } otherwise
	\end{cases}
	\end{equation}
	The amount label is calculated by,
	\begin{equation}
	amount=\left | price_{tomorrow} - price_{today} \right |
	\end{equation}
	and after the above calculation, the amount value will be normalized using standard normalization algorithm. 
\end{itemize}

After the above process, data is transformed and can be used for model training.


\section{Training and Predict}
As described before, there are two learning method used in this dissertation. 
\begin{itemize}
	\item Using one single algorithm to predict the stock price directly. Such algorithms include Linear Regression, Random Forest and ANN
	\item A combination of two learning method, using one algorithm to predict change direction (Logistic Regression, Random Forest and SVM), another to forecast amount (Linear Regression, Random, Forest and ANN). The two method works separately, and the predict result is the combination of both model.
\end{itemize}

On each iteration, system would randomly subsample the origin training dataset to get a different training set (a.k.a. bootstrapping). If the training algorithm does not support re-training (like Random Forest and SVM in pyspark.mllib), then the best model would be picked. Otherwise, just re-train the former model.\\


After training, the system would get a model then uses it to predict from testing dataset.




\section{Model Evaluation}
Compare forecasting performance of different learning method is one of the most important task of prediction study. This section introduces the criteria used in this dissertation.\\


let $ \hat{p} $ be the predicted variable, $ p $ be the actual stock price, $ \epsilon = \hat{p} - p $ be the forecast error, and $ N $ be the number of total testing sample number. Popular evaluation measures used in this study includes\cite{poon2005practical},
\begin{itemize}
	\item \textit{Mean Error} (ME)
	\begin{equation}
	ME=\frac{1}{N} \sum_{i=1}^{N}\epsilon_i=\frac{1}{N} \sum_{i=1}^{N} (\hat{p}_i - p_i)
	\end{equation}
	
	\item \textit{Mean Square Error} (MSE)
	\begin{equation}
	MSE = \frac{1}{N} \sum_{i=1}^{N}\epsilon_i^2=\frac{1}{N} \sum_{i=1}^{N} (\hat{p}_i - p_i)^2
	\end{equation}
	
	\item \textit{Root Mean Square Error} (RMSE)
	\begin{equation}
	RMSE = \sqrt{\frac{1}{N} \sum_{i=1}^{N}\epsilon_i^2}=\sqrt{\frac{1}{N} \sum_{i=1}^{N} (\hat{p}_i - p_i)^2}
	\end{equation}
	
	\item \textit{Mean Absolute Error} (MAE)
	\begin{equation}
	MAE=\frac{1}{N} \sum_{i=1}^{N} \lvert \epsilon_i \rvert =\frac{1}{N} \sum_{i=1}^{N} \lvert \hat{p}_i - p_i \rvert
	\end{equation}
	
	\item \textit{Mean Absolute Percent Error} (MAPE)
	\begin{equation}
	MAPE=\frac{1}{N} \sum_{i=1}^{N} \frac{\lvert \epsilon_i \rvert}{p_i} =\frac{1}{N} \sum_{i=1}^{N} \frac{\lvert \hat{p}_i - p_i \rvert}{p_i}
	\end{equation}
	
	\item \textit{Heteroscedasticity-adjust Mean Square Error} (HMSE)
	\begin{equation}
	HMSE=\frac{1}{N} \sum_{i=1}^{N}[\frac{p_i}{\hat{p}_i}- 1]^2
	\end{equation}
\end{itemize}
The first four measurements is scaled by predicted volatility, which are suitable to compare the prediction of same stock during equal period. The later two can be used to compare every method as they represent percentage errors.\\


In addition to knowing the amount of changes measurement, the direction of prices is also very important. Here, corrected direction change (CDC)\cite{naeini2010stock} are used
\begin{equation}
CDC = \frac{\text{Number of Corrected Forecast Trend}}{N}
\end{equation}
A completely random prediction should have $ 50\% $ CDC, so for a reliable forecast, this value should be greater than $ 50\% $\\


This is no measurement that are perfect enough to compare all learning algorithms, all criteria would be considered as a whole.