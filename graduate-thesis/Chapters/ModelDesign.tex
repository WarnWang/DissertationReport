\chapter{System Design}
\label{ch:system}

The objective of this dissertation is to build an automatic system which can download raw data, do transformation, train model, and give prediction result. This system is based on Python, a Friendly power programming language. “Quandl”, “TA-Lib”, “SciPy”, “pySpark”, “SciPy-learner” are used for data collecting, mining analysis and handling times series data structure. The whole process can be found in figure~\ref{fg:system_model}.
\begin{figure}[h]
	\centering
	\includegraphics[width=0.6\textwidth]{Model/FlowChart}
	\caption{Flow Chart of the whole system}
	\label{fg:system_model}
\end{figure}

After collected all raw data, system would split it into two parts, training and testing (details about collected data is introduced in Chapter\ref{ch:market}). To make an convincing experiment, \emph{the date of all testing data is behind that of training data}, so that the system would have no knowledge of the testing data. Details about the training process and parameters is introduced in Chapter~\ref{ch:modelTrainingProcess}.
